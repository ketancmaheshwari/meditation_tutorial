\documentclass[hyperref={pdfpagelabels=false},12pt]{beamer}
\usepackage[utf8]{inputenc}
\usepackage{multirow}
\usepackage{wasysym}
\usepackage{listings}
%\usepackage{gensymb}
\usepackage{array}
\usepackage{times}
\usepackage{xcolor}
\usepackage{ulem}
\beamertemplatenavigationsymbolsempty
\setbeamertemplate{footline}{\hspace*{.5cm}\scriptsize{ 
\hspace*{50pt} \hfill\insertframenumber\hspace*{.5cm}}} 
\beamertemplateballitem
\usefonttheme{default}
\xdefinecolor{lavendar}{rgb}{0.8,0.6,1}
\xdefinecolor{darkgreen}{rgb}{0.11,0.64,0.22}
\xdefinecolor{navyblue}{rgb}{0.06,0.31,0.55}
\xdefinecolor{black}{rgb}{0,0,0}
\xdefinecolor{olive}{cmyk}{0.64,0,0.95,0.4}
\title[meditation]{Meditation: A Tutorial}
\author[K. Maheshwari]{Ketan Maheshwari}

\setbeamertemplate{frametitle}[default][center]
\setbeamercolor{frametitle}{fg=black}
\setbeamertemplate{itemize items}[circle]
\setbeamertemplate{enumerate items}[circle]
\setbeamercolor{itemize item}{fg=black}
\setbeamercolor{enumerate items}{fg=black}

\date{}
\parskip 0.2cm
\lstset{
  basicstyle = \ttfamily\scriptsize\color{black},%\bfseries
  keywordstyle = \color{red},%\bfseries
  keywordstyle = [2]\ttfamily\color{black},
  frame = single,
  captionpos=b,
  frame=single,
  tabsize=2,
 language=XML
}

\begin{document}
\frame{\titlepage}

\frame{
\frametitle{Agenda}
\begin{block}{}
\begin{itemize}
\item Disclaimer
\item Overview / Definition
\item Benefits
\item Myths
\item How To
\item Bottomline
\end{itemize}
\end{block}
}

\frame{
\frametitle{Disclaimer}
\begin{block}{}
\centering
None of the claims of meditation benefits I make are guaranteed and the effects will vary from person to person.

A list of references on scientific studies may be found at:

{\tiny https://raw.githubusercontent.com/ketancmaheshwari/meditation\_tutorial/master/refs.txt}
\end{block}
}

\frame{
\frametitle{Overview}
\begin{block}{}
\centering
Meditation is a technique that may be defined as being in a state of non-judgemental awareness of the present moment.

Meditation is a technique that when learnt becomes a skill to train ones mind.

Many types of meditations, we will focus on breath meditation.
\end{block}
}

\frame{
\frametitle{Benefits -- Physical}
\begin{block}{}
\centering
Immune function 

pain

hypertension

brain structure

inflammation
\end{block}
}

\frame{
\frametitle{Benefits -- Psychological}
\begin{block}{}
\centering
Cognition

memory

creativity

stress

anxiety

depression
\end{block}
}

\frame{
\frametitle{Benefits -- Social}
\begin{block}{}
\centering
Empathy

introspection

compassion

emotion regulation
\end{block}
}
 
\frame{
\frametitle{Benefits -- Productivity}
\begin{block}{}
\centering
Improved focus

attention to details

ability to multi-task
\end{block}
}

\frame{
\frametitle{Myths}
\begin{block}{}
\centering
Religious practice / ritual

One has to sit still for hours to have any benefits

Breathing exercise
\end{block}
}

\frame{
\frametitle{How to Meditate -- Overview}
\begin{block}{}
\centering
Two parts: 

Technique

Practice
\end{block}
}

\frame{
\frametitle{How to Meditate -- Technique}
\centering
Sit comfortably with a straight back, eyes closed

Observe with attention the breathing \textbf{sensation} at nostrils

mind will eventually wander -- notice when it does

Gently bring attention back to breathing sensation
}

\frame{
\frametitle{How to Meditate -- Remarks}
\begin{block}{}
  \centering
Mind Wandering is normal

Attention will improve with practice

Start with 10 minute session, increase gradually
\end{block}
}

\frame{
\frametitle{How to Meditate -- Practice}
  \centering
A daily practice

Comfortable location

At same time preferably

Cumulative benefits!
}

\frame{
\frametitle{Bottomline (My personal take)}
\begin{block}{}
\centering
Similar to swimming, one has to experience meditation to realize its nature and benefits

\end{block}
}

\frame{
\frametitle{Thank You! Questions?}
\centering
\textbf{\texttt{ketancmaheshwari@gmail.com}}\\
}
\end{document}

